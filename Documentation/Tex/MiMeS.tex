\documentclass[11pt,a4paper]{article}
\usepackage[utf8]{inputenc}
\usepackage{amsmath}
\usepackage{amsfonts}
\usepackage{amssymb}
\usepackage[left=2cm,right=2cm,top=2cm,bottom=2cm]{geometry}


\usepackage{slashed}
\usepackage{hyperref}
\usepackage{graphicx}
\usepackage{caption}
\usepackage{float}
\usepackage{subcaption}

\usepackage{minted}
\usepackage[dvipsnames]{xcolor}


\renewcommand{\theequation}{\arabic{section}.\arabic{equation}}



\usepackage{ifthen}
\usepackage{tikz}
\usepackage{xspace}


%Define color environments:

\newenvironment{DK}[1]{{\color{gray}Commend from D: #1}}

\newenvironment{DKnew}[1]{{\color{blue}NEW from D: #1}}

\newenvironment{DKres}[1]{{\color{BrickRed}RESTRUCTURED from D: #1}}


\definecolor{mediumjunglegreen}{rgb}{0.11, 0.21, 0.18}
\definecolor{bg}{HTML}{282828}



\newenvironment{cpp}{\usemintedstyle{xcode}\VerbatimEnvironment\begin{minted}[frame=lines,framesep=5mm,baselinestretch=1.2,fontsize=\footnotesize,linenos,mathescape]{cpp}}{\end{minted}} 

\newenvironment{py}{\usemintedstyle{xcode}\VerbatimEnvironment\begin{minted}[frame=lines,framesep=5mm,baselinestretch=1.2,fontsize=\footnotesize,linenos,mathescape]{py}}{\end{minted}} 

\newenvironment{bash}{\usemintedstyle{xcode}\VerbatimEnvironment\begin{minted}[frame=lines,framesep=5mm,baselinestretch=1.2,fontsize=\footnotesize,linenos,mathescape]{bash}}{\end{minted}} 


\newenvironment{run}[1]{``{\tt #1}"\xspace}


\newcommand{\dint}{  \displaystyle \int }
%%%%%%%%%%%%%%%%%%%%%%%%%%%%%%%%%%%%%%%%%%
\newcommand{\ie}{{\em i.e.}\xspace}
\newcommand{\eg}{{\em e.g.}\xspace}
\newcommand{\GeV}{{\rm GeV}\xspace}
\newcommand{\TeV}{{\rm TeV}\xspace}
\newcommand{\MeV}{{\rm MeV}\xspace}
\newcommand{\keV}{{\rm keV}\xspace}

\def\mimes{{\tt MiMeS}\xspace}


\newcommand{\CPP}{{\tt C++}\xspace}

\newcommand{\PY}{{\tt python}\xspace}
	
\newcommand{\JUPY}{{\tt jupyter}\xspace}

\newcommand{\rhs}{RHS\xspace}
\newcommand{\lhs}{LHS\xspace}


\newcommand{\geff}{ g_{\rm eff}{}\xspace}
\newcommand{\heff}{ h_{\rm eff}{}\xspace}
 
\newcommand{\Ham}{ \mathcal{H}\xspace}
 
\newcommand{\thetamax}{ \theta_{\rm max}{}\xspace}

 
\newcommand{\thetai}{ \theta_{\rm ini}{}\xspace}

\newcommand{\fa}{ f_{a}{}\xspace}
 
 
\newcommand{\Ti}{ T_{\rm ini}{}\xspace}
 
\newcommand{\ti}{ t_{\rm ini}{}\xspace}
 
\newcommand{\ui}{ u_{\rm ini}{}\xspace}
 
\newcommand{\ai}{ a_{\rm ini}{}\xspace}

\newcommand{\thetaosc}{ \theta_{\rm osc}{}\xspace}

\newcommand{\Tosc}{ T_{\rm osc}{}\xspace}

\newcommand{\tosc}{ t_{\rm osc}{}\xspace}


\newcommand{\uosc}{ u_{\rm osc}{}\xspace}

\newcommand{\aosc}{ a_{\rm osc}{}\xspace}

\newcommand{\Omegai}{ \Omega_{\rm ini}\xspace}
 
\newcommand{\ma}{ m_{a}{}\xspace}
\newcommand{\maT}{ \tilde m_{a}{}\xspace}
\newcommand{\Lint}{ \mathcal{L}_{\rm int}\xspace}



\newcommand{\vev}[1]{\langle #1 \rangle}
\newcommand{\Bvev}[1]{\Bigg\langle #1 \Bigg\rangle}
\newcommand{\bvev}[1]{\Big\langle #1 \Big\rangle}




\newcommand{\lrb}[1]{\left( #1 \right)}
\newcommand{\lrsb}[1]{\left[ #1 \right]}
\newcommand{\lrBigb}[1]{\Big( #1 \Big)}
\newcommand{\lrBigsb}[1]{\Big[ #1 \Big]}
\newcommand{\lrBiggb}[1]{\Bigg( #1 \Bigg)}
\newcommand{\lrBiggsb}[1]{\Bigg[ #1 \Bigg]}

\newcommand{\lrBigcb}[1]{\Big\{ #1 \Big\}}
\newcommand{\lrBiggcb}[1]{\Bigg\{ #1 \Bigg\}}
%%%%%%%%%%%%%%%%%%%%%%%%%%%%%%%%%%%%%%%%%

%%%%%%%%%%%%%%%%%%%%%%%%%%%%%%%%%%%%%%%%%%%%%%%%%%%--Begin_refs--%%%%%%%%%%%%%%%%%%%%%%%%%%%%%%%%%%%%%%%%%%%%%%%%%%%%%%%%%%%%%%%%%%%%%%
\newcounter{NumArgs}

%Define reference to an arbitrary number of equations (\eqs{label_1,label_2....,label_n} will show eqs. ref_1, ref_2, ..., and ref_n)
\newcommand{\eqs}[1]{\setcounter{NumArgs}{0}\foreach\i in{#1}{\stepcounter{NumArgs}}%
\ifthenelse{\equal{\theNumArgs}{1}}{eq.~(\ref{#1})}%
{\ifthenelse{\equal{\theNumArgs}{2}}%
{eqs.~\foreach\i[count=\q]in{#1}{\ifthenelse{\equal{\q}{\theNumArgs}}{and (\ref{\i})}{(\ref{\i})~}}}%
{eqs.~\foreach\i[count=\q]in{#1}{\ifthenelse{\equal{\q}{\theNumArgs}}{and (\ref{\i})}{(\ref{\i}),~}}}}}


%Define reference to an arbitrary number of equations (\Eqs{label_1,label_2....,label_n} will show Eqs. ref_1, ref_2, ..., and ref_n)
\newcommand{\Eqs}[1]{\setcounter{NumArgs}{0}\foreach\i in{#1}{\stepcounter{NumArgs}}%
\ifthenelse{\equal{\theNumArgs}{1}}{Eq.~(\ref{#1})}%
{\ifthenelse{\equal{\theNumArgs}{2}}%
{Eqs.~\foreach\i[count=\q]in{#1}{\ifthenelse{\equal{\q}{\theNumArgs}}{and (\ref{\i})}{(\ref{\i})~}}}%
{Eqs.~\foreach\i[count=\q]in{#1}{\ifthenelse{\equal{\q}{\theNumArgs}}{and (\ref{\i})}{(\ref{\i}),~}}}}}


%Define reference to an arbitrary number of labels (\REF{label_1,label_2....,label_n} will show ref_1, ref_2, ..., and ref_n)
\newcommand{\refs}[1]{\setcounter{NumArgs}{0}\foreach\i in{#1}{\stepcounter{NumArgs}}%
\ifthenelse{\equal{\theNumArgs}{1}}{(\ref{#1})}%
{\ifthenelse{\equal{\theNumArgs}{2}}%
{\foreach\i[count=\q]in{#1}{\ifthenelse{\equal{\q}{\theNumArgs}}{and (\ref{\i})}{(\ref{\i})~}}}%
{\foreach\i[count=\q]in{#1}{\ifthenelse{\equal{\q}{\theNumArgs}}{and (\ref{\i})}{(\ref{\i}),~}}}}}



%Define reference to an arbitrary number of figs (\Figs{label_1,label_2....,label_n} will show ref_1, ref_2, ..., and ref_n)
\newcommand{\Figs}[1]{\setcounter{NumArgs}{0}\foreach\i in{#1}{\stepcounter{NumArgs}}%
\ifthenelse{\equal{\theNumArgs}{1}}{Fig.~(\ref{#1})}%
{\ifthenelse{\equal{\theNumArgs}{2}}%
{Figs.~\foreach\i[count=\q]in{#1}{\ifthenelse{\equal{\q}{\theNumArgs}}{and (\ref{\i})}{(\ref{\i})~}}}%
{Figs.~\foreach\i[count=\q]in{#1}{\ifthenelse{\equal{\q}{\theNumArgs}}{and (\ref{\i})}{(\ref{\i}),~}}}}}




%Define reference to an arbitrary number of "general reference" (\Gen{message}{label_1,label_2....,label_n} will show message.(ref_1), (ref_2), ..., and (ref_n)
\newcommand{\Gen}[2]{\setcounter{NumArgs}{0}\foreach\i in{#2}{\stepcounter{NumArgs}}%
	\ifthenelse{\equal{\theNumArgs}{1}}{#1.~(\ref{#2})}%
	{\ifthenelse{\equal{\theNumArgs}{2}}%
		{#1.~\foreach\i[count=\q]in{#2}{\ifthenelse{\equal{\q}{\theNumArgs}}{and (\ref{\i})}{(\ref{\i})~}}}%
		{#1.~\foreach\i[count=\q]in{#2}{\ifthenelse{\equal{\q}{\theNumArgs}}{and (\ref{\i})}{(\ref{\i}),~}}}}}


%%%%%%%%%%%%%%%%%%%%%%%%%%%%%%%%%%%%%%%%%%%%%%%%%%%--End_refs--%%%%%%%%%%%%%%%%%%%%%%%%%%%%%%%%%%%%%%%%%%%%%%%%%%%%%%%%%%%%%%%%%%%%%%



%%%%%%%-----------------------Rules-----------------------%%%%%%%
% 1. Put any new macros in macros.tex.

% 2. Section should start with:
	%\section{This is a section}\label{sec:Intro}
	%\setcounter{equation}{0}

% 3. labels for Figs should start with fig:, for equations should start with eq:, for sections with sec:, etc.
%%%%%%%--------------------------------------------------%%%%%%%




\author{Karamitros Dimitrios}
\title{{\tt MiMeS}: Misalignment Mechanism Solver}
\begin{document}

\maketitle

\begin{abstract}
	We introduce a \CPP header-only library that solves the Axion equation of motion, \mimes.  
	\mimes makes no assumptions regarding the cosmology and the thermal mass of the axion, which allows the user 
	to consider various cosmological scenarios and axion-like models.
	Moreover, although written entirely in \CPP, \mimes comes with a convenient \PY interface, which does not require the
	user to write any code in \CPP.
\end{abstract}


\section{Introduction}\label{sec:intro}
\setcounter{equation}{0}

About the axion...

\section{Physics background}\label{sec:abundance}
\setcounter{equation}{0}
%
Although there are several works in the literature (such as~\cite{Chang:1998ys}) that can provide a insight on the solution of the axion equation of motion (EOM), in this section we define, derive, and discuss the various quantities we need, in order to understand how \mimes works in detail.

\paragraph{The EOM} %The axion field is written as 
%
%\begin{equation}
%	A  \equiv \fa \ \theta,
%	\label{eq:fa_def}
%\end{equation}
%%
%with $\fa$ the scale of the axion that determines the the scale at which the PQ symmetry breaks. 
%
The EOM in terms of the axion angle, $\theta$, is 
%
\begin{equation}
	\lrb{\dfrac{d^2}{d t^2} + 3 H(t) \ \dfrac{d}{d t} } \theta(t) + \maT^2(t) \ \sin \theta(t) = 0 \; ,
	\label{eq:eom}
\end{equation}
%
with $H(t)$ the Hubble parameter (determined by the cosmology), and $\maT(t)$  the time (temperature) dependent mass of the axion, with 
\begin{equation}
	\maT^2(T) = \dfrac{\chi(T)}{\fa^2}\;,
	\label{eq:axion_mass_def}
\end{equation} 
%
and $\chi$ a function of the temperature, and $\fa$ the scale at which the Peccei Quinn symmetry breaks \DK{ref??}. For the QCD axion, this has been calculated using lattice simulations in~\cite{Borsanyi:2016ksw}.~\footnote{The data provided by ref.~\cite{Borsanyi:2016ksw} are used by  \mimes out-of-the-box. However, the user is free to change them.}


\paragraph{Initial conditions}
%
\begin{figure}[h!]
	\includegraphics[width=1\textwidth]{figs/axion_mass.pdf}
	\caption{The mass of the axion as a function of the temperature for $\fa=10^{12}~\GeV$, using the data provided in ref.~\cite{Borsanyi:2016ksw}.}
	\label{fig:axion_mass}
\end{figure}
%
Assuming that the PQ symmetry breaks before inflation, the initial conditions (\ie at some $t=0$, after inflation) for the EOM is random. However, we note that $\maT \to 0$ (see \Figs{fig:axion_mass}) -- \ie $\maT \ll H$ -- at very early times. Therefore, after inflation, the EOM is simply
%
\begin{equation}
	\lrb{\dfrac{d^2}{d t^2} + 3 H(t) \ \dfrac{d}{d t}  } \theta(t) = 0 \; ,
	\label{eq:massless_eom}
\end{equation}
%
which is solved by $\theta = \thetai + C \dint_{0}^t d t' \ \lrb{ \dfrac{a(t'=0)}{a(t')} }^3$, with $\thetai$ a constant and $a$ the scale factor of the Universe. That is, as the Universe expands, $\theta \approx \thetai$. Since we would like to calculate the relic abundance of axions, we can integrate \eqs{eq:eom} from a time after inflation (call it $t = \ti$) such that $ \dot \theta |_{t=\ti} = 0$ and  $\theta|_{t=\ti} \approx \thetai$.   



\subsection{The WKB approximation}
%
In order to solve analytically \eqs{eq:eom}, we assume $\theta \ll 1$, which results in the linearised EOM
%
\begin{equation}
	\lrb{\dfrac{d^2}{d t^2} + 3 H(t) \ \dfrac{d}{d t} + \maT^2(t) } \theta(t) = 0 \; .
	\label{eq:linear_eom}
\end{equation}

Using a trial solution $\theta_{\rm trial} = \exp\lrsb{ i \dint d t \ \lrBigb{u(t) +3/2 \ i \ H(t)} }$, and defining $\Omega^2 = \maT^2 - \dfrac{9}{4} H^2 -  \dfrac{3}{2} \dot H $ we can transform the \eqs{eq:linear_eom} to 
%
\begin{equation}
	u^2 = \Omega^2 + i \ \dot u \; ,
	\label{eq:eom_of_u}
\end{equation}
%
which has a formal solution $u = \pm \sqrt{\Omega^2 + i \dot u}$. Assuming that $\dot u \ll \Omega^2$ and $\dot \Omega \ll \Omega^2$, we can approximate $u$ as
%
\begin{equation}
	u \approx \pm \Omega + \dfrac{i}{2} \dfrac{d \log \Omega}{d t} \;,
	\label{eq:u_approx}
\end{equation}
%
which results in the general solution of \eqs{eq:linear_eom} 
%
\begin{equation}
	\theta \approx \dfrac{1}{\sqrt{\Omega}} \exp\lrb{-\dfrac{3}{2} \int d t \ H} \lrsb{ A \cos\lrb{ \int d t \ \Omega} +  B \sin\lrb{ \int d t \ \Omega}    } \;. 
	\label{eq:general_solution_eom_approx}
\end{equation}

Applying, then, the initial conditions $ \dot \theta |_{t=\ti} = 0$ and  $\theta|_{t=\ti} \approx \thetai$, we arrive at 
%
\begin{equation}
\theta(t) \approx \thetai \sqrt{ \dfrac{ \Omegai }{\Omega (t)} } \lrb{\dfrac{a}{\ai}}^{-3/2} \  \cos\lrb{ \int_{\ti}^t d t^\prime  \ \Omega(t^\prime)}   \;.
\label{eq:solution_eom_approx} 
\end{equation}


In order to further simplify this approximate result, we note that $\theta$ deviates from $\thetai$ close to $t=\tosc$ -- corresponding to $T = \Tosc$, the so-called ``oscillation temperature" -- $\maT|_{t = \tosc} = 3 H|_{t = \tosc}$, which is defined as the point at which the axion begins to oscillate. 
%
This observation allows us to set $\ti = \tosc$.  Moreover, at $t > \tosc$, we approximate $\Omega \approx \maT$, as $H^2$ and $\dot H$ become much smaller than $\maT^2$ quickly after $t=\tosc$. Finally, the axion angle takes the form
%
\begin{equation}
	\theta(t) \approx \thetaosc \lrb{\dfrac{3}{4}}^{1/4} \sqrt{ \dfrac{ \maT|_{t=\tosc} }{\maT  (t)} } \lrb{\dfrac{a}{\aosc}}^{-3/2} \  \cos\lrb{ \int_{\tosc}^t d t^\prime  \ \maT(t^\prime)}   \;,
	\label{eq:solution_eom_approx_theta_osc} 
\end{equation}
%
where $\thetaosc = \theta|_{t=\tosc}$. This equation is further simplified if we assume that $\thetaosc \approx \thetai$, \ie
%
\begin{equation}
	\theta(t) \approx \thetai \lrb{\dfrac{3}{4}}^{1/4} \sqrt{ \dfrac{ \maT|_{t=\tosc} }{\maT  (t)} } \lrb{\dfrac{a}{\aosc}}^{-3/2} \  \cos\lrb{ \int_{\tosc}^t d t^\prime  \ \maT(t^\prime)}   \;.
	\label{eq:solution_eom_approx_final} 
\end{equation}
%
It is worth mentioning that the accuracy of this approximation depends, in general, on $\Tosc$; it determines the difference between $\thetai$ and $\thetaosc$, the deviation of $\dot \theta|_{t=\tosc}$ from $0$, and whether $\dot \Omega \ll \Omega^2$. 


\paragraph{Axion energy density}
%
The energy density of the axion is 
%
\begin{eqnarray}
	\rho_{a} = \dfrac{1}{2} \fa^2 \lrsb{ \dot{\theta}^2 + \maT^2 \theta^2 } \;.
	\label{eq:rho_a_def} 
\end{eqnarray}
%
For the relic abundance of axions, we need to calculate their energy density at very late times. That is, $\dot{\tilde{m}}_a = 0$, $\maT \gg H$ and $\dot H \ll H^2$. After some algebra, we obtain the approximate form of the energy density (as a function of the scale factor $a$) 
%
\begin{eqnarray}
	\rho_{a} \approx \dfrac{\ma }{2}  \ \fa^2 \ \thetai^2  \ \maT(\aosc) \ \lrb{\dfrac{\aosc}{a}}^3 \;,
	\label{eq:rho_a0} 
\end{eqnarray}
%
which shows that the energy density of axions at late times scales as the energy density of matter. If there is a period of entropy injection to the plasma for $T<\Tosc$, the axion energy density gets diluted, since 
%
\begin{equation}
	a^3 \ s = \gamma \ \aosc^3 \ s_{\rm osc} \Rightarrow  \lrb{\dfrac{\aosc}{a}}^3 = \gamma^{-1} \dfrac{s}{s_{\rm osc}} \;,
\end{equation}
%
with $\gamma$ the amount of entropy injection to the plasma between $\tosc$ and $t$. Therefore, the present (at $T=T_0$) energy density of the axion, becomes
%
\begin{eqnarray}
	\rho_{a,0} = \gamma^{-1}  \dfrac{s_0}{s_{\rm osc}} \  \dfrac{1 }{2}  \ \fa^2 \ \ma \ \maT_{,{\rm osc}} \ \thetai^2    \;,
	\label{eq:rho_a_approx} 
\end{eqnarray}
with $\ma$ the mass of the axion at $T=T_0$. Notice that the explicit dependence on $\fa$ cancels, since $\maT \sim 1/\fa$. That is, $\fa$ only affects the energy density of the axions through its impact on $\Tosc$. 
%

\subsection{Notation}\label{sec:notation}
%
%The WKB approximation is very useful in order to understand the evolution of the axion field. However, it fails to explain how the oscillation begins before $\dot \Omega \ll \Omega^2$ is reached. In this section we will try to understand the evolution of the axion as generally as possible. 
%
The EOM~(\ref{eq:eom}) depends on time, which is not very useful variable in a non-standard comsological setting. Therefore, we introduce 
%
\begin{eqnarray}
	u = \log \dfrac{a}{\ai} \;,
	\label{eq:natation}
\end{eqnarray}
%
which results in 
%
\begin{eqnarray}
	&\dfrac{d F}{dt} &=  H  \dfrac{d F}{du} 
	\nonumber \\
	&\dfrac{d^2 F}{dt^2} &= H^2 \ \lrb{ \dfrac{d^2 F}{du^2} + \dfrac{1}{2} \dfrac{d \log H^2}{du}  \dfrac{d F}{du} }\;.
	\label{eq:deriv_u}
\end{eqnarray}

The EOM in terms of $u$, then, becomes
%
\begin{equation}
	\dfrac{d^2  \theta}{du^2} + \lrsb{\dfrac{1}{2} \dfrac{d \log H^2}{du} + 3 } \dfrac{d  \theta}{d u} + \ \lrb{\dfrac{\maT}{H}}^2 \ \sin \theta
	=0 \;.
	\label{eq:eom_u}
\end{equation}

Notice that in a radiation dominated Universe
%
$$
\dfrac{d \log H^2}{du} = -\lrb{ \dfrac{d \log \geff}{d \log T} +4 } \delta_h^{-1}\;,
$$
with  $ \delta_h = 1+ \dfrac{1}{3} \dfrac{d \log \heff}{d \log T} $. 
%
In a general cosmological setting, the expansion rate is dominated by an energy density that scales as $\rho \sim a^{-c}$, which results in $\dfrac{d \log H^2}{du}  = -c$. However, close to rapid particle annihilations and decays, the evolution of the energy densities change, and $\dfrac{d \log H^2}{du}$ can only be computed numerically.

\subsection{Beyond the WKB approximation}\label{sec:beyond_WKB}
%
The WKB approximation is very useful, as it helps us understand the evolution of the axion field after it begins to oscillate adiabatically. However, it fails to capture the dynamics the adiabatic conditions are met, result in inaccurate axion relic abundance result. In this section, we examine the deviation of $\thetaosc$ from the initial value of $\theta$, as well how to treat cases where the EOM cannot be linearized.

\subsubsection{Behaviour close to the initial condition}\label{sec:close_to_ini}
%
The accuracy of the approximate evolution of the axion angle given in \eqs{eq:solution_eom_approx_final} depends on the deviation between $\thetai$ and $\thetaosc$. In order to examine their deviation, we expand \eqs{eq:eom} at a time $t =\ti + \delta t$ with $\delta t \to 0$. That is, we use the following approximations 
%
\begin{equation*}
	\ddot{\theta} \approx \dfrac{\dot \theta (\ti+ \delta t)  - \dot \theta (\ti)}{ \delta  t } =  \dfrac{\dot \theta (\ti + \delta t)  }{\delta  t } \;,
	\qquad
	\dot \theta(\ti + \delta t) = \dfrac{\theta(\ti + \delta t) - \thetai}{\delta t} \;.
\end{equation*} 
%
and, solve the EOM~\ref{eq:eom} for $\theta(\ti + \delta t)$. This results in
%
\begin{equation}
	\theta(\ti + \delta t)  \approx  \thetai -\dfrac{\delta t^2}{1+3\ \delta t \ H(\ti + \delta t) } \  \maT^2(\ti + \delta t)  \sin \theta(\ti + \delta t)  
	\approx   \thetai - \delta t^2 \ \maT^2(\ti) \ \sin \thetai  \ + \mathcal{O}(\delta t^3)\;, 
	\label{eq:theta_dt}
\end{equation}
%
which indicates that the angle decreases (increases) as the temperature drops if $\thetai>0$ ($\thetai<0$). Using the notation introduced in section~\ref{sec:notation}, \eqs{eq:theta_dt} takes the form
%
\begin{eqnarray}
	\theta \approx    \thetai - \delta u^2 \ \lrb{\dfrac{\maT}{H}}_{t=\ti}^2 \ \sin \thetai \;,
	\label{eq:theta_du}
\end{eqnarray}
%
which can be used to estimate the angle at the oscillation temperature
%
\begin{eqnarray}
	\thetaosc \approx    \thetai -  \lrb{\dfrac{\maT}{H}}_{t=\ti}^2 
	\lrsb{1- \dfrac{\Ti}{\Tosc}\lrb{\dfrac{\heff_{,\rm ini}}{\heff_{,\rm osc}}\gamma_{\rm osc}}^{1/3}}^{2}   \ \sin \thetai \;,
	\label{eq:theta_osc}
\end{eqnarray}
%
where  $\gamma_{\rm osc}$ is the entropy injection between $\Ti$ and $\Tosc$. Notice that in the derivation of \eqs{eq:solution_eom_approx_final} we used $\theta_{\rm osc} = \thetai$ as our first approximation. Thus, \eqs{eq:theta_osc} provides a correction that takes into account the deviation between $\theta_{\rm osc} $ and $ \thetai$, and \eqs{eq:rho_a_approx} becomes (for $\thetai \ll 1$)
%
\begin{eqnarray}
	\rho_{a,0} = \gamma^{-1}  \dfrac{s_0}{s_{\rm osc}} \  \dfrac{1 }{2}  \ \fa^2 \ \ma \ \maT_{,{\rm osc}} \ \thetai^2 \lrBiggcb{
		1 - \ \lrb{\dfrac{\maT}{H}}_{t=\ti}^2 \  \lrb{1- \dfrac{\Ti}{\Tosc}\lrb{\dfrac{\heff_{,\rm ini}}{\heff_{,\rm osc}}\gamma_{\rm osc}}^{1/3}}^{2}   }^2    \;,
	\label{eq:rho_a_NLO} 
\end{eqnarray}
%
which implies that the WKB approximation overestimates the energy density of the axion, since $\gamma_{\rm osc}$, $H_{\rm ini}$, and $\Tosc$ can vary. 


In order to demonstrate the effect of the deviation between $\thetai$ and $\thetaosc$, we show \Figs{fig:WKB_diff_i,fig:WKB_diff_osc}, for different cosmological scenarios. In both figures, the red and black (dashed) lines correspond to radiation and matter dominated Universe, respectively.~\footnote{For the matter dominated case (following the notation of refs.~\cite{Arias:2019uol,Arias:2020qty}), we have used 
$T_{\rm end}=10^{-2} ~\GeV,\; c=3, \; T_{\rm ini}=10^{12} ~\GeV, \; \text{and}\; r=0.1$.} For the matter dominated case, entropy is injected to the plasma both before and after $\Tosc$. The entropy injection factor, $\gamma_{\rm osc}$, as a function of $\fa$ is shown in \Figs{fig:gamma_osc}.

In \Figs{fig:WKB_diff_i} we show the relative difference between $\Omega h^2$ using \eqs{eq:solution_eom_approx_final} and the result obtained from \mimes. It should be apparent that the accuracy of the estimate based on \eqs{eq:solution_eom_approx_final} depends of $\fa$, since for high values of $\fa$, $\Tosc$ is low. That is, the deviation between $\thetai$ and $\thetaosc$ is greater than the corresponding difference for lower $\fa$, especially for the matter dominated case, where the entropy injection is great (see \Figs{fig:gamma_osc}).

In \Figs{fig:WKB_diff_osc} we show the relative difference between $\Omega h^2$ obtained by integrating numerically \eqs{eq:eom_u}  and using \eqs{eq:solution_eom_approx_theta_osc} with \eqs{eq:theta_osc}. The relative difference between the two estimates, once again, depends $\fa$. It should be noted that the estimate for $\thetaosc$ depends on the choice of $\Ti$. For \Figs{fig:WKB_diff_osc}, we define $\Ti$ as the temperature at which $3H/\maT = 2$. We note that this estimate is more accurate, its accuracy still depends on whether the axion evolves adiabatically at $T<\Tosc$. Moreover, an increase of $\gamma_{\rm osc}$ invalidates the estimate of $\thetaosc$; higher powers of $\delta u$ may be needed.    
%
\begin{figure}[t]
	\begin{subfigure}[]{0.5\textwidth}
		\includegraphics[width=1\textwidth]{figs/WKB_diff_i.pdf}
		\caption{}
		\label{fig:WKB_diff_i}
	\end{subfigure}
	\begin{subfigure}[]{0.5\textwidth}
		\includegraphics[width=1\textwidth]{figs/WKB_diff_osc.pdf}
		\caption{}
		\label{fig:WKB_diff_osc}
	\end{subfigure}
	\begin{center}
	\begin{subfigure}[]{0.5\textwidth}
		\includegraphics[width=1\textwidth]{figs/gamma_osc.pdf}
		\caption{}
		\label{fig:gamma_osc}
	\end{subfigure}
	\end{center}
\caption{...}
\label{fig:WKB_diff}
\end{figure}
%

Nevertheless, neither estimates based on \eqs{eq:solution_eom_approx_theta_osc,eq:solution_eom_approx_final} seem to provide a result comparable to numerically integrating \eqs{eq:eom_u}. Therefore, numerical integration should be preferred. 

\subsubsection{Adiabatic invariant and the anharmonic factor}\label{sec:an_fac}
%
In oscillatory systems with varying period, the energy is not conserved, and it is usually useful to define an ``adiabatic invariant", which is an approximate constant of motion.

\paragraph{Definition of the adiabatic invariant}
%
Given a system with Hamiltonian $\mathcal{H}(\theta,p;t)$, the equations of motion are 
%
\begin{equation}
	\dot p = - \dfrac{\partial \mathcal{H}}{\partial \theta} \;, \;\; 
	\dot \theta =  \dfrac{\partial \mathcal{H}}{\partial p} \;.
	\label{eq:hamiltonian_eoms}
\end{equation}

Moreover, we note that
%
\begin{equation}
	d \Ham = \dot \theta \ d p - \dot p \ d \theta + \dfrac{\partial \Ham}{\partial t} \ d t \;.  
	\label{eq:total_dH}
\end{equation}


If this system exhibits closed orbits (\eg if it oscillates), we define 
%
\begin{equation}
	J \equiv C \ \oint p \ d \theta \;,
	\label{eq:adiabatic_inv_def}
\end{equation}
%
where the integral is over a closed path (\eg a period, $T$), and $C$ indicates that $J$ can always be rescaled with a constant. This quantity is the adiabatic invariant of the system, if the Hamiltonian varies slowly during a cycle. That is,
%
\[
\dfrac{d J}{d t} = C \ \oint \lrBigb{\dot p \ d \theta + p \ d \dot \theta} = C \ \dint_{t}^{t+T}  \dfrac{\partial \Ham}{\partial t^\prime} \ d t^\prime \approx T \ \dfrac{\partial \Ham(t^{\prime})}{\partial t^{\prime}}\Big|_{t^{\prime}=t} \approx 0 
\;. 
\]
%




\paragraph{Application to the axion}
%
The Hamiltonian that results in the EOM of \eqs{eq:eom} is
%
\begin{equation}
	\Ham = \dfrac{1}{2} \dfrac{p^2}{\fa^2 \ a^3} + V(\theta) \ a^3\;,
	\label{eq:axion_H}
\end{equation}
%
with 
%
\begin{eqnarray}
	& p = \fa^2 \ a^3 \ \dot \theta \\
	\label{eq:momentum}
	& V(\theta) = \maT^2 \fa^2 (1-\cos \theta) \;.
	\label{eq:potential}
\end{eqnarray}

Notice that the Hamiltonian varies slowly if $\dot \maT/\maT \ll \maT$ and $H \ll \maT$, which are the adiabatic conditions.  When these conditions are met, the adiabatic invariant for this system becomes
%
\begin{equation}
	J = \dfrac{\oint p \ d \theta}{\pi \fa^2} = \dfrac{1}{\pi \fa^2} \oint \sqrt{ 2\lrb{ \Ham(\theta) - V(\theta) \ a^3} \ \fa^2 a^3 \ }  \ d \theta  =
	 \dfrac{2}{\pi \fa^2} \int_{-\thetamax}^{\thetamax} \sqrt{ 2\lrb{ \Ham(\thetamax) - V(\theta) \ a^3} \ \fa^2 a^3 \ } d \theta \;,
	 \label{eq:J_axion_definition}
\end{equation}
%
where we note that $\thetamax$ denotes the maximum of $\theta$ -- the peak of the oscillation, which corresponds to $p=0$. That is, $\Ham(\thetamax) = V(\thetamax) \ a^3$. Therefore, the adiabatic invariant, takes the form 

\begin{eqnarray}
	J=&  \dfrac{2 \sqrt{2} }{\pi \fa}  \int_{- \thetamax} ^{\thetamax}  \sqrt{ V(\theta_{\rm max}) - V(\theta) } a^{3} d \theta = 
	\dfrac{2 \sqrt{2} }{\pi} \ \maT \, a^3 \ \dint_{- \thetamax}^{\thetamax} \sqrt{\cos \theta - \cos \thetamax} \ d \theta  
	\;,
	\label{eq:J_axion_derivation}
\end{eqnarray}
%
where, for the last equality. we have used the adiabatic conditions, \ie negligible change of $\maT$ and $a$ during one period. Usually, the adiabatic invariant is written as~\cite{Lyth:1991ub,Bae:2008ue} 
%
\begin{equation}
	J = a^3 \ \maT \ \thetamax^2  \, f(\thetamax)  \;,
	\label{eq:J_axion_final_form}
\end{equation}
%
where 
\begin{equation}
	f(\thetamax) =\dfrac{ 2 \sqrt{2}}{\pi \thetamax^2 } \dint_{- \thetamax}^{\thetamax} d \theta \sqrt{ \cos \theta - \cos \thetamax } \;,
	\label{eq:anharmonic_f}
\end{equation}
%
is called the anharmonic factor, with $ 0.5 \lesssim f(\thetamax) \leq 1$ (see \Figs{fig:anharmonic_factor}).


\begin{figure}[t]
	\includegraphics[width=1\textwidth]{figs/anharmonic_factor.pdf}
	\caption{The anharmonic factor for $0 \leq \thetamax < \pi $.}
	\label{fig:anharmonic_factor}
\end{figure}



\paragraph{The role of the adiabatic invariant in the axion relic energy density}
%
The adiabatic invariant allows us to calculate the maximum value of the angle $\theta$ at late times from its corresponding value at some point after the adiabatic conditions where met. It allows us to calculate the energy density of the axions field for $\theta \gtrsim 1$ since we have taken into account the exact potential.

In order to do this, we can numerically integrate \eqs{eq:eom}, and identify the maxima of $\theta$. Once the adiabatic conditions are fulfilled, we can stop the integration at a peak, $\thetamax_{,*}$ -- which corresponds to $T=T_{*}$ and $a=a_{*}$. Then, the value of the maximum angle today ($\thetamax_{,0} \ll 1$) is related to $\thetamax_{,*}$ via
%
\begin{eqnarray}
	\thetamax_{,0}^2 &=  \lrb{\dfrac{a_*}{a_0}}^3 \ \dfrac{\maT_{,*}}{\ma} \ f(\thetamax_{,*}) \ \thetamax_{,*}^2  =
	\gamma^{-1} \ \dfrac{s_0}{s_*} \ \dfrac{\maT_{,*}}{\ma} \ f(\thetamax_{,*}) \ \thetamax_{,*}^2 
	\; .
	\label{eq:theta_relation}
\end{eqnarray}
%
Using this, and since we evaluate the energy density at the maximum of $\theta$ (\ie $\dot \theta = 0$), we can determine the energy density of axions today from \eqs{eq:rho_a_def}. That is,
%
\begin{equation}
	\rho_{a,0} = \gamma^{-1} \ \dfrac{s_0}{s_*} \ \ma \ \maT_{,*} \ \dfrac{1}{2} \ \fa^2 \ \thetamax_{,*}^2 \;  \ f(\thetamax_{,*}) \;.
	\label{eq:rho_axion_exact}
\end{equation}
%
Notice that this is form of $\rho_{a,0}$  is similar to the the WKB result~(\ref{eq:rho_a_approx}) at $\tosc \to t_*$, multiplied by the anharmonic factor $f(\thetamax_{,*})$. That is, if the axion starts to evolve adiabatically close to $t=\tosc$, and $\thetaosc \ll 1$, the WKB approximation is valid.






\section{How to start using \mimes}\label{sec:start}
\setcounter{equation}{0}
%
\begin{listing}
	\begin{bash}
		git clone git@github.com:dkaramit/MiMeS.git
		cd MiMeS
		git submodule init
		git submodule update --remote
	\end{bash}
	%
	\caption{Commands in order to download \mimes, the differential equation solvers, and the interpolation libraries.}
	\label{list:git_download}
\end{listing}
%
The library can downloaded from \href{https://github.com/dkaramit/MiMeS}{github.com/dkaramit/MiMeS}. Since the library depends on {\tt NaBBODES}~\cite{NaBBODES} {\tt SpleSplines}~\cite{SimpleSplines}, one has to download the, too. However, in order to get \mimes ready, one can run the commands shown in listing~\ref{list:git_download}. Although this method should be preferred in order to get the latest version, \mimes can also be downloaded from \href{https://mimes.hepforge.org/downloads/}{mimes.hepforge.org/downloads}.


Once everything is downloaded successfully, we can go inside the \mimes directory, and run {\tt bash configure.sh}~\footnote{Instead of {\tt bash}, shells such {\tt sh} should also work.} and {\tt make}.  The {\tt bash} script {\tt configure.sh}, just writes some paths to some files, formats the data files provided in a acceptable format (in section~\ref{sec:input} the format is explained), and makes some directories. The {\tt makefile} is responsible for compiling some examples and checks, as well as the shared libraries that needed for the \PY interface.  If everything runs successfully, there should be two new directories {\tt exec} and {\tt lib}. Inside {\tt exec}, there are several executables that ready to run, in order to ensure that the code runs (\eg no segmentation fault occurs). For example, {\tt exec/AxionSolve\_check.run}, should print the values of the parameters $\thetai$ and $\fa$, the oscillation temperature and the corresponding value of $\theta$, the evolution of the axion (\eg temperature, $\theta$, $\rho_{a}$, etc.), and the values of various quantities on the peaks of the oscillation.  In the directory {\tt lib}, there are several shared libraries for the \PY interface.

\subsection{First steps} There are several examples in \CPP ({\tt UserSpace/Cpp}) and \PY ({\tt UserSpace/Python}), as well as \JUPY  notebooks ({UserSpace/JupyterNotebooks}), that show in detail how \mimes can be used. 

\subsubsection{Using \mimes in \CPP} 
%
In order to use the {\tt mimes::Axion} class in a \CPP program, we need ti include the header file {\tt AxionSolve.hpp} located in {\tt src/Axion}. That is, on top of the {\tt .cpp}, we need to write 
%
\begin{cpp}
	#include "src/Axion/AxionSolve.hpp"
\end{cpp}
%
Notice that if the  {\tt .cpp} is not in the root directory of \mimes, we need to compile it using the flag {\tt -Ipath-to-root}, "path-to-root" the relative path to the root directory of \mimes; \eg if the {\tt .cpp} is in the {\tt UserSpace/Cpp/Axion} directory, this flag should be {\tt -I../../../}.

Then, we can assign an instance of {\tt mimes::Axion} in a variable using
%
\begin{cpp}
	mimes::Axion ax(theta_i, fa, umax, TSTOP, ratio_ini, N_convergence_max,convergence_lim,
	inputFile, initial_step_size,minimum_step_size, maximum_step_size, absolute_tolerance, 
	relative_tolerance, beta, fac_max, fac_min, maximum_No_steps);
\end{cpp}
%
The various parameters are as follows:
%
\begin{enumerate}
	\item {\tt theta\_i}: initial angle.
	\item {\tt fa}: the PQ scale.
	\item {\tt umax }: if $u>${\tt umax} the integration stops (remember that $u=\log(a/a_i)$). Typically, this should be a large number ($\sim 1000$), in order to avoid stopping the integration before the axion begins to evolve  adiabatically.    
	\item {\tt TSTOP}: if the temperature drops below this, integration stops. In most cases this should be around 
	$10^{-4}~\GeV$, in order to be sure that any entropy injection has stopped before integration stops (since BBN bounds~\cite{Kolb:206230,Peebles:1993} should not be violated).
	\item {\tt ratio\_ini}: integration starts when $3H/\maT \approx${\tt ratio\_ini} (the exact point depends on the file ``{\tt inputFile}", which we will see later). 
	\item  {\tt N\_convergence\_max} and {\tt convergence\_lim}: integration stops when the relative difference 
	between two consecutive peaks is less than {\tt convergence\_lim} for {\tt N\_convergence\_max} 
	consecutive peaks.
	\item  {\tt inputFile}: relative (or absolute) path to a file that describes the cosmology. the columns should be: $u$ $T ~[\GeV]$ $\log H$, sorted so that $u$ increases.~\footnote{One can run {\tt bash src/FormatFile.sh inputFile} in order to sort it and remove any unwanted duplicates. See Appendix~\ref{app:util} for details of {\tt src/FormatFile.sh}.}
	%
	It is important to remember that \mimes assumes that the entropy injection has stopped before the lowest temperature of given in {\tt inputFile}. Since \mimes is unable to guess the cosmology beyond what is given in this file, the user has to make sure that there are data between the initial temperature (which corresponds to {\tt ratio\_ini}, and {\tt TSTOP}).
	
	\item {\tt initial\_stepsize} (optional): initial step the solver takes. 
	
	\item {\tt maximum\_stepsize} (optional): This limits the step-size to an upper limit. 
	\item {\tt minimum\_stepsize} (optional): This limits the step-size to a lower limit. 
	
	\item {\tt absolute\_tolerance} (optional): absolute tolerance of the RK solver
	
	\item {\tt relative\_tolerance} (optional): relative tolerance of the RK solver.
	%
	Generally, both absolute and relative tolerances should be $10^{-8}$. 
	In some cases, however, one may need more accurate result (\eg if {\tt f\_a} is extremely high, 
	the oscillations happen violently, and the system destabilizes). Whatever the case, if the  
	tolerances are below $10^{-8}$, long doubles have be used. \mimes by default uses {long double} variables, 
	in order to change it see the options available in section~\ref{sec:input}.
	
	\item {\tt beta} (optional): controls how agreesive the adaptation is. Generally, it should be around but less than 1.
	
	\item {\tt fac\_max},  {\tt fac\_min} (optional): the stepsize does not increase more than fac\_max, and less than fac\_min. 
	This ensures a better stability. Ideally, {\tt fac\_max}$=\infty$ and {\tt fac\_min}$=0$, but in reality one must 
	tweak them in order to avoid instabilities.
	
	\item {\tt maximum\_No\_steps} (optional): maximum steps the solver can take Quits if this number is reached even if integration
	is not finished. 
\end{enumerate}
%
in order to understand the usage of the optional parameters, some basic techniques of Runge-Kutta methods are discussed in Appendix~\ref{app:RK}. 

The EOM~(\ref{eq:eom_u}), then can be solved using 
%
\begin{cpp}
	ax.solveAxion()
\end{cpp}
%
This, if this runs successfully (in simple cases, such as a radiation dominated Universe, this should take a fraction of a second), then  $\Tosc$, $\thetaosc$, and $\Omega h^2$ are accessed via {\tt ax.T\_osc}, {\tt ax.theta\_osc}, and {\tt ax.relic}. The entire evolution (the points the integrator took) of the axion angle is stored in {\tt ax.points}, which is a two-dimensional {\tt std::vector}, with the columns corresponding to  $a$, $T~[\GeV]$, 
$\theta$, $d\theta/du$, $\rho_a$. Moreover, the peaks of he oscillation are stored in another two-dimensional {\tt std::vector}, with the columns corresponding to $a$, $T~[\GeV]$, $\thetamax$, $d\theta/du=0$, $\rho_a$, $J$. We should note that the peaks are identified using linear interpolation between integration points, in order to ensure that $d\theta/du = 0$. That is, the values stored in {\tt mimes::Axion::peaks} do not exist in {\tt mimes::Axion::points}.


\subsubsection{Using \mimes in \PY} The modules for the \PY interface are located in {\tt src/interfacePy}. Since, the most useful module is the one that actually solves the EOM~\ref{eq:eom_u}, we will explain here it can be used, and describe the others in Appendix~\ref{app:modules}.

The {\tt Axion} class in the module {\tt interfacePy.Axion} can be loaded in a \PY script as 
%
\begin{py}
	from sys import path as sysPath
	from os import path as osPath
	sysPath.append(osPath.join(osPath.dirname(__file__), 'path_to_src'))
	from interfacePy.Axion import Axion
\end{py}
%
It is important that \mintinline{python}{'path_to_src'} provides the relative path to the {\tt src} directory. For example, if the script is located in {\tt UserSpace/Python}, \mintinline{python}{'path_to_src'} should be \mintinline{python}{'../../src'}.

Once the module is imported, we can define an {\tt Axion} instance as follows 
%
\begin{py}
	ax=Axion(theta_i, fa, umax, TSTOP, ratio_ini, N_convergence_max, convergence_lim, inputFile,
	initial_step_size,minimum_step_size, maximum_step_size, absolute_tolerance, 
	relative_tolerance, beta, fac_max, fac_min, maximum_No_steps)
\end{py}
%
Here, {\tt Axion} is the constructor of the {\tt Axion} class, and {\tt ax} the variable -- which can have any name, and the input parameters are exactly the same as in the \CPP case (the usage of the class can be found by running {\tt ?Axion} after loading the module). 

Using the defined variable ({\tt ax} in this example), we can simply run  
%
\begin{py}
	ax.solveAxion()
\end{py}
%
in order to solve the EOM of the axion. In contrast to the \CPP implementation, this only gives us access to $\Tosc$, $\thetaosc$, and $\Omega h^2$; the corresponding variables are {\tt ax.T\_osc}, {\tt ax.theta\_osc}, and {\tt ax.relic}. In order to get the evolution of the axion field, we need to run 
%
\begin{py}
	ax.getPoints()
\end{py}
%
This will make {\tt numpy} arrays that contain the scale factor ({\tt ax.a}), temperature ({\tt ax.T}), $\theta$ ({\tt ax.theta}), its derivative with respect to $u$ ({\tt ax.zeta}), and the energy density of the axion ({\tt ax.rho\_axion}).

Moreover, in order to get the various quantities on the peaks of the oscillation, we can run
%
\begin{py}
	ax.getPeaks()
\end{py}
%
This makes {\tt numpy} arrays that contain the scale factor ({\tt ax.a\_peak}), temperature ({\tt ax.T\_peak}), $\theta$ ({\tt ax.theta\_peak}), its derivative with respect to $u$ ({\tt ax.zeta\_peak}, which should be equal to $0$), the energy density of the axion ({\tt ax.rho\_axion\_peak}), and the values of the adiabatic invariant on the peaks ({\tt ax.adiabatic\_invariant}).

\paragraph{Importand} Th {\tt Axion} class constructs a {\tt mimes::Axion} pointer, which has to be manually deleted. Therefore, once {\tt ax} is used it must be deleted, \ie we need to run 
%
\begin{py}
	del ax
\end{py}
%
We should note that this must run even if we assign another instance to the same variable {\tt ax}, otherwise we risk running out of memory.

\section{Assumptions and user input}\label{sec:assumptions}
\setcounter{equation}{0}
%
\mimes only makes a few, fairly general, assumptions. First of all, it is assumed that the axion energy density is always subdominant compared to radiation or any other component of the Universe, and that decays and annihilations of particles have a negligible effect on the axion energy density. Moreover, the initial condition is always assumed to be $\theta_{t=\ti} = \thetai$ and $\dot \theta|_{t=\ti}=0$. Moreover, it is also assumed that $3H/\maT$ decreases monotonically at high temperatures. 

The user can provide optional input at compile-time (\eg change the Runge-Kutta method to be used, or provide files with data for the relativistic degree of freedom of the plasma), but is required to provide some input at run-time, in order for \mimes to run. 

\subsection{User input}\label{sec:input}
%
\subsubsection{Required files} \mimes requires files that provide data for the relativistic degrees of freedom (RDOF) of the plasma, as well as its mass as a function of the temperature. Although \mimes is shipped with the standard model RDOF found in~\cite{Saikawa:2020swg}, and the axion mass from~\cite{Borsanyi:2016ksw}, the user can change the corresponding variables in {\tt Definitions.mk} (found in the root directory of \mimes) to another file path. The variables pointing to these data files are {\tt cosmoDat} and {\tt axMDat}, for the RDOF and axion mass, respectively.~\footnote{One can change the data file for anharmonic factor introduced in \eqs{eq:anharmonic_f} using the variable {\tt anFDat}.}

The format of the files is the following:~\footnote{The anharmonic factor data are given in $\thetamax$ $f(\thetamax)$, with increasing $\thetamax$.} 
%
\begin{itemize}
	\item The RDOF data must be given in three columns; $T ~[\GeV]$, $\heff$, and $\geff$, and must be sorted so that $T$ is increasing.
	\item The axion mass data must be given in two columns; $T ~[\GeV]$, $\chi$, sorted so that $T$ is increasing. Here, $\chi$ is defined as in \eqs{eq:axion_mass_def}. 
\end{itemize}
%
The paths to these files should be set at compile time. That is, once {\tt Definitions.mk} changes, one has to run {\tt bash configure.sh} and then {\tt make}. The user can change the content of the data files (without changing their paths), in order to use them without compiling \mimes.

\subsubsection{Required input}

\DK{Since you have described them, talk about why you need them. For example say that {\tt umax} should not be omitted because \mimes is designed to be as general as possible.}


\subsection{Complete Examples}\label{sec:complete_examples}

\paragraph{A complete \CPP example}\DK{Give the code with explanation of {\tt UserSpace/CPP/Axion}, along with the compilation options.}

\paragraph{A complete \PY example}\DK{Give th/e code with explanation of {\tt UserSpace/Python/Axion.py}, say that the compilation options are set in the root directory of \mimes.}

\DK{Mention optimization options, and {\tt g++} and {\tt clang} seem to work. Describe the available solvers (appendix for more details).}.


	%%%%%%%%%%%%%%%%%%%%%%%%%%%%%%%%%%%%%%%%
% APPENDICES
%\pagebreak
%%%%%%%%%%%%%%%%%%%%%%%%%%%%%%%%%%%%%%%%
\setcounter{section}{0}
\section*{Appendix}
\appendix

\renewcommand{\theequation}{\Alph{section}.\arabic{equation}}
\setcounter{equation}{0}  % reset counter
%%%%%%%%%%%%%%%%%%%%%%%

\section{Basics of embedded Runge-Kutta Mehtods}\label{app:RK}
\setcounter{equation}{0}
\DK{Basically, take a few things you have written in ASAP and explain them (give a link to ASAP, in case someone wants to take a  closer look).}
\paragraph{Explicit methods}{\DK{Show the basic equations and butcher tableu}}
\paragraph{Rosenbrock methods}{\DK{Show the basic equations and butcher tableu}}

\section{\CPP classes}\label{app:classes}
\setcounter{equation}{0}
\DK{Describe all the classes.}

\section{\PY modules}\label{app:modules}
\setcounter{equation}{0}
\DK{Describe the modules.}



\section{Utilities that make running \mimes easier}\label{app:util}
\setcounter{equation}{0}
\DK{Describe {\tt FormatFile.sh} and {\tt Axion.sh}.}

\section{Quick guide to user input}\label{app:usr_input}
\setcounter{equation}{0}
\DK{make a table with all the variables one can change}


%%%%%%%%%%%%%%%%%%%%%%%%%%%%%%%
\newpage
\bibliography{refs}{}
\bibliographystyle{JHEP}                        

\end{document}
