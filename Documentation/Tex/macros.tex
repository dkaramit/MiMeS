\usepackage{ifthen}
\usepackage{tikz}
\usepackage{xspace}


%Define color environments:

\newenvironment{DK}[1]{{\color{gray}Commend from D: #1}}

\newenvironment{DKnew}[1]{{\color{blue}NEW from D: #1}}

\newenvironment{DKres}[1]{{\color{BrickRed}RESTRUCTURED from D: #1}}


\definecolor{mediumjunglegreen}{rgb}{0.11, 0.21, 0.18}
\definecolor{bg}{HTML}{282828}



\newenvironment{cpp}{\usemintedstyle{xcode}\VerbatimEnvironment\begin{minted}[frame=lines,framesep=5mm,baselinestretch=1.2,fontsize=\footnotesize,linenos,mathescape]{cpp}}{\end{minted}} 

\newenvironment{py}{\usemintedstyle{xcode}\VerbatimEnvironment\begin{minted}[frame=lines,framesep=5mm,baselinestretch=1.2,fontsize=\footnotesize,linenos,mathescape]{py}}{\end{minted}} 

\newenvironment{bash}{\usemintedstyle{xcode}\VerbatimEnvironment\begin{minted}[frame=lines,framesep=5mm,baselinestretch=1.2,fontsize=\footnotesize,linenos,mathescape]{bash}}{\end{minted}} 

%in order to writ maths outside of comments put them inside ?$$?. Theta is for x^2, write ?$x^2$?.
\newenvironment{pseudo}{\usemintedstyle{xcode}\VerbatimEnvironment\begin{minted}[frame=lines,framesep=5mm,baselinestretch=1.2,fontsize=\footnotesize,linenos,escapeinside=??,mathescape=true]{python}}{\end{minted}} 

\newenvironment{run}[1]{``{\tt #1}"\xspace}


\newcommand{\dint}{  \displaystyle \int }
%%%%%%%%%%%%%%%%%%%%%%%%%%%%%%%%%%%%%%%%%%
\newcommand{\ie}{{\em i.e.}\xspace}
\newcommand{\eg}{{\em e.g.}\xspace}
\newcommand{\GeV}{{\rm GeV}\xspace}
\newcommand{\TeV}{{\rm TeV}\xspace}
\newcommand{\MeV}{{\rm MeV}\xspace}
\newcommand{\keV}{{\rm keV}\xspace}

\def\mimes{{\tt MiMeS}\xspace}


\newcommand{\CPP}{{\tt C++}\xspace}

\newcommand{\PY}{{\tt python}\xspace}
	
\newcommand{\JUPY}{{\tt jupyter}\xspace}

\newcommand{\rhs}{RHS\xspace}
\newcommand{\lhs}{LHS\xspace}


\newcommand{\geff}{ g_{\rm eff}{}\xspace}
\newcommand{\heff}{ h_{\rm eff}{}\xspace}
 
\newcommand{\Ham}{ \mathcal{H}\xspace}
 
\newcommand{\thetamax}{ \theta_{\rm peak}{}\xspace}

 
\newcommand{\thetai}{ \theta_{\rm ini}{}\xspace}

\newcommand{\fa}{ f_{a}{}\xspace}
 
 
\newcommand{\Ti}{ T_{\rm ini}{}\xspace}
 
\newcommand{\ti}{ t_{\rm ini}{}\xspace}
 
\newcommand{\ui}{ u_{\rm ini}{}\xspace}
 
\newcommand{\ai}{ a_{\rm ini}{}\xspace}

\newcommand{\thetaosc}{ \theta_{\rm osc}{}\xspace}

\newcommand{\Tosc}{ T_{\rm osc}{}\xspace}

\newcommand{\tosc}{ t_{\rm osc}{}\xspace}


\newcommand{\uosc}{ u_{\rm osc}{}\xspace}

\newcommand{\aosc}{ a_{\rm osc}{}\xspace}

\newcommand{\Omegai}{ \Omega_{\rm ini}\xspace}
 
\newcommand{\ma}{ m_{a}{}\xspace}
\newcommand{\maT}{ \tilde m_{a}{}\xspace}
\newcommand{\Lint}{ \mathcal{L}_{\rm int}\xspace}



\newcommand{\vev}[1]{\langle #1 \rangle}
\newcommand{\Bvev}[1]{\Bigg\langle #1 \Bigg\rangle}
\newcommand{\bvev}[1]{\Big\langle #1 \Big\rangle}




\newcommand{\lrb}[1]{\left( #1 \right)}
\newcommand{\lrsb}[1]{\left[ #1 \right]}
\newcommand{\lrBigb}[1]{\Big( #1 \Big)}
\newcommand{\lrBigsb}[1]{\Big[ #1 \Big]}
\newcommand{\lrBiggb}[1]{\Bigg( #1 \Bigg)}
\newcommand{\lrBiggsb}[1]{\Bigg[ #1 \Bigg]}

\newcommand{\lrBigcb}[1]{\Big\{ #1 \Big\}}
\newcommand{\lrBiggcb}[1]{\Bigg\{ #1 \Bigg\}}
%%%%%%%%%%%%%%%%%%%%%%%%%%%%%%%%%%%%%%%%%

%%%%%%%%%%%%%%%%%%%%%%%%%%%%%%%%%%%%%%%%%%%%%%%%%%%--Begin_refs--%%%%%%%%%%%%%%%%%%%%%%%%%%%%%%%%%%%%%%%%%%%%%%%%%%%%%%%%%%%%%%%%%%%%%%
\newcounter{NumArgs}

%Define reference to an arbitrary number of equations (\eqs{label_1,label_2....,label_n} will show eqs. ref_1, ref_2, ..., and ref_n)
\newcommand{\eqs}[1]{\setcounter{NumArgs}{0}\foreach\i in{#1}{\stepcounter{NumArgs}}%
\ifthenelse{\equal{\theNumArgs}{1}}{eq.~(\ref{#1})}%
{\ifthenelse{\equal{\theNumArgs}{2}}%
{eqs.~\foreach\i[count=\q]in{#1}{\ifthenelse{\equal{\q}{\theNumArgs}}{and (\ref{\i})}{(\ref{\i})~}}}%
{eqs.~\foreach\i[count=\q]in{#1}{\ifthenelse{\equal{\q}{\theNumArgs}}{and (\ref{\i})}{(\ref{\i}),~}}}}}


%Define reference to an arbitrary number of equations (\Eqs{label_1,label_2....,label_n} will show Eqs. ref_1, ref_2, ..., and ref_n)
\newcommand{\Eqs}[1]{\setcounter{NumArgs}{0}\foreach\i in{#1}{\stepcounter{NumArgs}}%
\ifthenelse{\equal{\theNumArgs}{1}}{Eq.~(\ref{#1})}%
{\ifthenelse{\equal{\theNumArgs}{2}}%
{Eqs.~\foreach\i[count=\q]in{#1}{\ifthenelse{\equal{\q}{\theNumArgs}}{and (\ref{\i})}{(\ref{\i})~}}}%
{Eqs.~\foreach\i[count=\q]in{#1}{\ifthenelse{\equal{\q}{\theNumArgs}}{and (\ref{\i})}{(\ref{\i}),~}}}}}


%Define reference to an arbitrary number of labels (\REF{label_1,label_2....,label_n} will show ref_1, ref_2, ..., and ref_n)
\newcommand{\refs}[1]{\setcounter{NumArgs}{0}\foreach\i in{#1}{\stepcounter{NumArgs}}%
\ifthenelse{\equal{\theNumArgs}{1}}{(\ref{#1})}%
{\ifthenelse{\equal{\theNumArgs}{2}}%
{\foreach\i[count=\q]in{#1}{\ifthenelse{\equal{\q}{\theNumArgs}}{and (\ref{\i})}{(\ref{\i})~}}}%
{\foreach\i[count=\q]in{#1}{\ifthenelse{\equal{\q}{\theNumArgs}}{and (\ref{\i})}{(\ref{\i}),~}}}}}



%Define reference to an arbitrary number of figs (\Figs{label_1,label_2....,label_n} will show ref_1, ref_2, ..., and ref_n)
\newcommand{\Figs}[1]{\setcounter{NumArgs}{0}\foreach\i in{#1}{\stepcounter{NumArgs}}%
\ifthenelse{\equal{\theNumArgs}{1}}{Fig.~(\ref{#1})}%
{\ifthenelse{\equal{\theNumArgs}{2}}%
{Figs.~\foreach\i[count=\q]in{#1}{\ifthenelse{\equal{\q}{\theNumArgs}}{and (\ref{\i})}{(\ref{\i})~}}}%
{Figs.~\foreach\i[count=\q]in{#1}{\ifthenelse{\equal{\q}{\theNumArgs}}{and (\ref{\i})}{(\ref{\i}),~}}}}}




%Define reference to an arbitrary number of "general reference" (\Gen{message}{label_1,label_2....,label_n} will show message.(ref_1), (ref_2), ..., and (ref_n)
\newcommand{\Gen}[2]{\setcounter{NumArgs}{0}\foreach\i in{#2}{\stepcounter{NumArgs}}%
	\ifthenelse{\equal{\theNumArgs}{1}}{#1.~(\ref{#2})}%
	{\ifthenelse{\equal{\theNumArgs}{2}}%
		{#1.~\foreach\i[count=\q]in{#2}{\ifthenelse{\equal{\q}{\theNumArgs}}{and (\ref{\i})}{(\ref{\i})~}}}%
		{#1.~\foreach\i[count=\q]in{#2}{\ifthenelse{\equal{\q}{\theNumArgs}}{and (\ref{\i})}{(\ref{\i}),~}}}}}


%%%%%%%%%%%%%%%%%%%%%%%%%%%%%%%%%%%%%%%%%%%%%%%%%%%--End_refs--%%%%%%%%%%%%%%%%%%%%%%%%%%%%%%%%%%%%%%%%%%%%%%%%%%%%%%%%%%%%%%%%%%%%%%